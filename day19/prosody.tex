\documentclass[pdf]{beamer}
\mode<presentation>{}
\usetheme{Rochester}
\usecolortheme{whale}
\beamertemplatenavigationsymbolsempty

\usepackage[ngerman]{babel}
\usepackage{fontspec}
\setmainfont[ItalicFont={brilli.ttf}, BoldFont={brillb.ttf}, BoldItalicFont={brillbi.ttf}]{brill.ttf}
\usefonttheme{serif}
\usepackage{microtype}
\usepackage{multicol}

\newcommand{\Subitem}[1]{{\setlength\itemindent{12pt} \item[-] #1}}
\newcommand{\Subsubitem}[1]{{\setlength\itemindent{24pt} \item[○] #1}}
\newcommand{\Subsubsubitem}[1]{{\setlength\itemindent{36pt} \item[-] #1}}
\newcommand{\Subsubsubsubitem}[1]{{\setlength\itemindent{48pt} \item[○] #1}}

\title{Meeting 19.5: Sanskrit prosody}
\subtitle{Examining some basic meters}
\author{Nikhil Surya Dwibhashyam}
\date{24 July 2022}

\begin{document}

\frame{\titlepage}

\begin{frame} \frametitle{Light and heavy syllables}
\begin{itemize}
	\item Indigenous vs.~western method of splitting syllables
	\Subitem Comes out to the same syllable count and quantity
	\item Long vowel?
	\Subitem Heavy syllable
	\item Consonant at end (western)? Cluster beginning next (Indian)?
	\Subitem Heavy syllable
	\item Generally: Heavy (–) twice as long as light (⏑)
\end{itemize}
\end{frame}

\begin{frame} \frametitle{Indo-European metrical tradition}
\begin{itemize}
	\item Quantitative (but not in all branches)
	\item Diiambic cadenza
\end{itemize}
\begin{center}
	⏑–⏑– | ⏑–⏑–

	Evolution of Greek hexameter:

	–⏑⏑– | ⏑⏑––
\end{center}
\end{frame}

\begin{frame} \frametitle{Common meters}
\begin{center}
	Gāyatrá and anuṣṭúbʰ:

	XXXX⏑–⏑–

	Epic anuṣṭúbʰ:

	XXXX⏑––– | XXXX⏑–⏑– 

	Examples of triṣṭúbʰ (catalexis of jágatī):

	XXXX • ⏑⏑––⏑––

	XXXXX • ⏑⏑–⏑––
\end{center}
\end{frame}

\begin{frame} \frametitle{Common meters}
\begin{center}
	Poëm from today:

	Kurmahe padya-saṁ-rambʰaṁ Nikʰil’-Āgni-pra-cālitam
	
	Pālikā-ṡabda-vidy’-āugʰa-pālitaṁ vi-jayiṣyate.
	
	Ketu-Praṇava-bʰūyiṣṭaṁ kurmahe kāvya-saṁ-graham
	
	ṡataṁ samāṡ calet prāyo Vidʰinā vihitam param.
\end{center}
\end{frame}

\end{document}
